\documentclass[12pt]{article}
\usepackage{a4wide}
\usepackage[utf8]{inputenc}
\usepackage[T1]{fontenc}
\usepackage[francais]{babel}
\usepackage{times}
\usepackage{graphicx}
\usepackage{url}
\usepackage[pdftex,colorlinks=true,pdfstartview=FitV,linkcolor=blue,citecolor=blue,urlcolor=blue]{hyperref}
\usepackage{xspace}   

\newcommand{\tab}{\hspace*{1cm}}
\newcommand{\comment}[1]{\textcolor{red}{\bf\em #1}\par}

\begin{document}
\vspace*{-3.5cm}
\hspace*{-2.5cm}
\begin{tabular}{lc}
\includegraphics{logoMinesDouai-small} 
& 
\begin{minipage}[b]{15cm}
\begin{center}
{\it TP Programmation par objet}\\~\\
{\Large\bf Concurrence}\\~\\
{\sf Noury Bouraqadi}
\end{center}
\end{minipage}\\
\end{tabular}

\newcommand{\st}[1]{{\small \verb!#1!}} 


\section{Thread et affichage textuel}
Vous allez utiliser le \sf{Transcript} dans cet exercice et analyser le texte qui y apparaitra. 
Afin de vous simplifiez la vie, effacez son contenu avant chaque exécution.
Il est possible de faire cela directement dans votre code à l'aide de l'expression \st{Transcript clear}.
 
\subsection{Threads simples}
Définissez deux threads en faisant appel à la méthode \st{fork} et qui tous les deux effectuent un affichage le \st{Transcript}.
\begin{itemize}
	\item le premier affiche 10 fois la chaîne de caractères \st{'ping'} précédée d'un saut de ligne;
	\item le second affiche 10 fois la chaîne de caractères \st{'PONG'} précédée d'un saut de ligne.
\end{itemize} 

Quel affichage obtenez-vous ? Expliquez.

\subsection{Priorités de threads}
Modifiez vos threads de manière à leur attribuer des priorités différentes :
\begin{itemize}
	\item attribuez la priorité 30 au premier thread (qui affiche 'ping');
	\item attribuez la priorité 40 au second thread (qui affiche 'PONG')
\end{itemize}

Quel affichage obtenez-vous ? Expliquez.

\subsection{Attentes}
Ajoutés des délais d'attentes dans vos threads :
\begin{itemize}
	\item attendez 100 millisecondes avant chaque affichage pour le premier thread (qui affiche 'ping');
	\item attendez 300 millisecondes avant chaque affichage pour le second thread (qui affiche 'PONG').	
\end{itemize}

Quel affichage obtenez-vous ? Expliquez.


\section{Balles Rebondissantes}
Soit la classe \st{SimulationBalleRebondissante} fournie qui correspond à une simulation de balle rebondissante.
Elle met en oeuvre un \emph{thread} qui contrôle le mouvement d'une balle.
Elle répond aux méthodes :
\begin{itemize}
	\item \st{lancerSimulation} affiche une balle rebondissante et lance le thread qui en contrôle le déplacement.
	\item \st{arreterSimulation} arrête le thread qui contrôle le déplacement de la balle.
	\item \st{arreterAffichage} masque l'affichage de la balle rebondissante.
\end{itemize}

\begin{enumerate}
	\item Testez le bon fonctionnement d'une simulation. 
	\item Lancez une simulation et sauvegardez votre image et Pharo \textbf{sans} quitter la simulation. Relancez votre image Pharo. Que constatez-vous
	\item Construire, puis lancez simultanement une collection de 100 simulations. Que constatez-vous ? 
	\item Proposez une solution qui résout le problème.
\end{enumerate}


%%%%%%%%%% 
\section{Course d'escargots}
Vous allez réaliser dans cet exercice une course d'escargots. 
Les escargots seront seront affichés sous forme de graphique à l'aide de la classe \st{AfficheurEscargot fournie}.
Vous utiliserez le \emph{pattern} observateur pour actualiser l'affichage.

La course est chronométrée.
Votre application doit avoir un thread pour le chronometre, ainsi qu'un thread par escargot.
Ainsi, une course de 3 escargot nécessitera 4 threads.
Lorsque le premier escargot a depassé la ligne d'arrivé, tous les threads doivent être arrêtés et la valeur du chronomètre doit être affichée sur le \st{Transcript}.

Les escargots sont initialement alignés sur la gauche de l'écran et doivent parcourir une distance equivalente à plusieurs fois leurs longeur (un escargot fait 150 pixels de long).
Un escargot avance d'un pas (10 pixels) à chaque pas de temps (100 millisecondes).
Il dispose d'une quantité d'énergie qui correspond au nombre de pas consécutifs qu'il peut effectuer.
A chaque pas, le niveau d'énergie est décrémenté.

Lorsque le niveau d'énergie atteint 0 l'escargot rentre dans sa coquille et dors pendant un nombre de pas de temps aléatoire (entre 1 et 10).
Pendant le sommeil, le niveau d'énergie est incrémenté d'une valeur aléatoire entre 1 et 10.
Lorsque l'escargot se réveille, il reprend la course.

A sa création, la quantité d'energie initiale d'un escargot est définie de manière aléatoire entre 1 et 10.
Pour le tirage aléatoire, vous utiliserez la méthode \st{atRandom} permet d'obtenir un élément au hasard d'une collection.
Par exemple l'expression \st{(25 to: 43) atRandom} retourne un élément au hasard dans l'intervalle de nombres entiers dont les bornes sont 25 et 43 inclus.

\end{document}