\documentclass[12pt]{article}
\usepackage{a4wide}
\usepackage[utf8]{inputenc}
\usepackage[T1]{fontenc}
\usepackage[francais]{babel}
\usepackage{times}
\usepackage{graphicx}
\usepackage{url}
\usepackage[pdftex,colorlinks=true,pdfstartview=FitV,linkcolor=blue,citecolor=blue,urlcolor=blue]{hyperref}
\usepackage{xspace}   

\begin{document}
\vspace*{-3.5cm}
\hspace*{-2.5cm}
\begin{tabular}{lc}
\resizebox{2cm}{!}{\includegraphics{logoMinesDouai-small}} 
& 
\begin{minipage}[b]{15cm}
\begin{center}
{\it TP Programmation par Objets}\\~\\
{\Large\bf Collections et Héritage : Tetris}\\~\\
{\sf Noury Bouraqadi}
\end{center}
\end{minipage}\\
\end{tabular}


\newcommand{\st}[1]{{\small \verb!#1!}} 

\section*{Description du jeu}
L'objectif du TP est développer un jeu de tetris simple, où des pièces composées d'un ou de plusieurs carrés tombent.
Vous devez les déplacer avec les touches flèches du clavier pour les déplacer vers la gauche ou vers la droite.
Le but est d'atteindre le score maximum en plaçant le plus de pièces possibles.

\section*{Code fourni}
Vous utiliserez l'image fournie.
Elle contient des classes finalisées pour notamment gérer l'affichage et le déroulement du jeu.
La classe principale est \st{TsTetris}. 
Elle permet de lancer le jeu en évaluant dans un workspace l'expression \st{TsTetris start}.

\section*{Travail à réaliser}
Complétez le jeu en ajoutant des classes manquantes à savoir :
\begin{itemize}
	\item \st{TsCarre} : C'est un fragment de pièce. Il occupe une case complète dans le terrain.
	\item \st{TsBarre} : C'est la classes des pièces rectilignes horizontales composées chacune de 1 à 4 carré.
	\item \st{TsTerrain} : Représente le terrain dans le quel tombent les pièces. Un terrain dispose d'une matrice de cases qui peuvent être vides (contenu égal à \st{nil}) ou pleines (le contenu est un carré).
\end{itemize}


\end{document}